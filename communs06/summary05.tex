\documentclass{article}
  % Packages and settings
  \usepackage{fontspec}
    \setmainfont{Charis SIL}
  \usepackage[margin=1.0in]{geometry}
  \usepackage[style=apa, backend=biber]{biblatex}
    \addbibresource{References.bib}

  % Document information
  \title{Summary of Podesva (2007)}
  \author{Joshua McNeill}
  \date{29 October 2020}

  % New commands
  \newcommand{\orth}[1]{$\langle$#1$\rangle$}
  \newcommand{\lexi}[1]{\textit{#1}}
  \newcommand{\gloss}[1]{`#1'}

\begin{document}
  \maketitle
  \fullcitebib{podesva_phonation_2007}
  \section{Introduction and The Study}
    \noindent{}\citeauthor{podesva_phonation_2007}'s (\citeyear{podesva_phonation_2007}) objective with this article is to show that variable phonation types, particularly falsetto, can be used to construct personae.
    \begin{itemize}
      \item To do this, he analyzes the speech of Heath, a gay medical student, in various situations.
    \end{itemize}
    Work on phonation, especially in sociolinguistics, has been scarce, but there have been some notable examples.
    \begin{itemize}
      \item Most work has used impressionistic analyses by trained phoneticians and focused on associations with communities and groups.
    \end{itemize}
  \section{Falsetto and Creaky Voice}
    Falsetto is defined as lengthening and adducting of the vocal folds, leading to higher F0 values.
  \section{Gay Identity}
  \section{Discussion}
\end{document}
