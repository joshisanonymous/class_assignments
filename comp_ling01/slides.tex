\documentclass{beamer}
  % Beamer settings
  \usetheme{Berkeley}
  \usecolortheme{dove}
  \usecolortheme{rose}
  \usefonttheme{professionalfonts}
  \usefonttheme{serif}

  % Packages and settings
  \usepackage{fontspec}
    \setmainfont{Charis SIL}
  \usepackage{hyperref}
    \hypersetup{
      colorlinks=true,
      allcolors=blue
    }
  % \usepackage[style=apa, backend=biber]{biblatex}
  %   \addbibresource{References.bib}

  % Document information
  \author{Joshua McNeill}
  \title{Summary of Eisenstein and R̆ehůr̆ek \& Kolkus (2009)}
  \date{7 October 2020}

\begin{document}
  \begin{frame}
    \titlepage
  \end{frame}

  \begin{frame}
    \tableofcontents[hideallsubsections]
  \end{frame}

  \AtBeginSection[]{
    \begin{frame}
      \tableofcontents[
        currentsection,
        hideallsubsections
      ]
    \end{frame}
  }

  \section{Language Models}
    \newcommand{\subone}{What are they?}
    \subsection{\subone}
      \begin{frame}{\subone}
        \begin{block}{The probability of a document given some tokens}
          \begin{itemize}
            \item $p(w_1, w_2, \ldots, w_m)$
            \item $w_m \in V$
            \item $V = \{ \text{aardvark}, \text{abacus}, \ldots, \text{zither} \}$
          \end{itemize}
        \end{block}
      \end{frame}

      \begin{frame}{\subone}
        \begin{block}{What good is this?}
          Lets us generate strings
          {\small
            \begin{itemize}
              \item $p ( \text{coffee} | \text{I love dark} )$
            \end{itemize}
          }
          Lets us compare strings
          {\small
            \begin{itemize}
              \item $p ( \text{the coffee black me pleases much} ) < p ( \text{I love dark coffee} )$
            \end{itemize}
          }
        \end{block}
      \end{frame}

    \newcommand{\subtwo}{$n$-gram models}
    \subsection{\subtwo}
      \begin{frame}{\subtwo}
        \begin{block}{A relative frequency approach}
          $p ( \text{Computers are useless, they only give you answers} )$
          \begin{itemize}
            \item[=] $\frac{count(\text{Computers are useless, they only give you answers})}
                        {count(\text{all sentences ever spoken})} $
          \end{itemize}
        \end{block}
        \begin{alertblock}{}
          Impossible to give the latter count
        \end{alertblock}
      \end{frame}

      \begin{frame}{\subtwo}
        \begin{block}{The chain rule}
          {\small
            \begin{itemize}
              \item[] $p ( \text{string} )$
              \item[] $= p ( w_1, w_2, \ldots, w_m )$
              \item[] $= p ( w_1 ) \times p ( w_2 | w_1 ) \times p ( w_3 | w_2, w_1 ) \times \ldots \times p ( w_m | w_{m-1}, \ldots, w_1 )$
            \end{itemize}
          }
        \end{block}

        \begin{block}{Simplified into an $n$-gram model}
          {\small
            \begin{itemize}
              \item[] $\prod_m^M{ p ( w_m | w_{m-1}, \ldots, w_{m-n+1} ) }$
            \end{itemize}
          }
        \end{block}
      \end{frame}

      \begin{frame}{\subtwo}
        \begin{block}{Bigram for ``I like black coffee''}
          {\small
            \begin{itemize}
              \item[] $p ( \text{I} | \# ) \times p ( \text{like} | \text{I} ) \times p ( \text{black} | \text{like} ) \times p ( \text{coffee} | \text{black} )$
            \end{itemize}
          }
        \end{block}

        \begin{block}{}
          Each bigram's individual probability is estimated from the relative frequency in some text
        \end{block}
      \end{frame}

      \begin{frame}{\subtwo}
        \begin{block}{$n$-gram sizes}
          Too small:
          \begin{itemize}
            \item Long distance dependencies are lost
            \item ``\alert{Gorillas} always like to groom \alert{their} friends''
          \end{itemize}
          Too big:
          \begin{itemize}
            \item Data is too sparse to get meaningful counts
            \item How many times does the gorilla 7-gram show up?
          \end{itemize}
        \end{block}
      \end{frame}

    \newcommand{\subthree}{Smoothing}
    \subsection{\subthree}
      \begin{frame}{\subthree}
        \begin{block}{A solution to sparse data}
          Laplace smoothing: Add an imaginary count of 1 to all counts
          \begin{itemize}
            \item e.g., bigrams
            \item[] $p_{\text{smooth}} ( w_m | w_{m-1} )$
            \item[=] $\frac{count ( w_{m-1}, w_m ) + 1}
                           {\sum_{w' \in V}{count ( w_{m-1}, w' ) + 1}}$
          \end{itemize}
        \end{block}

        \begin{block}{}
          Add a different imaginary count to get a different type of smoothing
          \begin{itemize}
            \item Jeffreys-Perks law: 0.5
          \end{itemize}
        \end{block}
      \end{frame}

      \begin{frame}{\subthree}
        \begin{block}{Discounting}
          
        \end{block}
      \end{frame}

  \section{Language Classification}
\end{document}
