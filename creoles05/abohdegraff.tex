\documentclass{article}
  % Packages and settings
  \usepackage{fontspec}
    \setmainfont{Charis SIL}
  \usepackage{setspace}
  \usepackage[style=apa, backend=biber]{biblatex}
    \addbibresource{References.bib}
  % \usepackage{graphicx}
  %   \graphicspath{{figures/}}

  % Document information
  \title{Summary of Aboh \& deGraff (2017)}
  \author{Joshua McNeill}
  \date{\today}

  % New commands
  \newcommand{\orth}[1]{$\langle$#1$\rangle$}
  \newcommand{\lexi}[1]{\textit{#1}}
  \newcommand{\gloss}[1]{`#1'}
  \newcommand{\titl}[1]{``#1''}

\begin{document}
  \maketitle
  \onehalfspacing
    \citeauthor{roberts_null_2017}'s (\citeyear{roberts_null_2017})

    % Start by giving a history of the term creole
      % appeared first in the 16th century for New World flora, fauna, and humans
      % compare the appearance of creoles to the development of romance languages after rome fell
      % cover how early scholars promoted creole exceptionalism and the racialization of creoles
        % two ideas
          % creoles are linked to Creoles who are in a middle position in racial hierarchy
          % creoles are simpler than other languages (so that simple minds can use them)
        % incompatible with UG
        % they also argue against the idea that there's no geneological link between HC and French by comparing the preverbal markers to paraphrastic French constructions
    \printbibliography
\end{document}
