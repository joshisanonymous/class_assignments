\documentclass{article}
  % Packages and settings
  \usepackage{fontspec}
    \setmainfont{Charis SIL}
  \usepackage{setspace}
  \usepackage[style=apa, backend=biber]{biblatex}
    \addbibresource{References.bib}
  \usepackage{graphicx}
    \graphicspath{{figures/}}

  % Document information
  \title{Reflection 1: Ethnolects in Louisiana}
  \author{Joshua McNeill}
  \date{\today}

  % New commands
  \newcommand{\orth}[1]{$\langle$#1$\rangle$}
  \newcommand{\lexi}[1]{\textit{#1}}
  \newcommand{\gloss}[1]{`#1'}
  \newcommand{\image}[1]{
    \begin{tabular}[t]{c}
      \\
      \includegraphics[scale=0.7]{#1}
    \end{tabular}
  }

\begin{document}
  \maketitle
  \doublespacing
  \section{Ethnolects}
    Ethnolects are language varieties that are associated with ethnicities.
    Often these varieties are named for the ethnicities they are associated with, such as Chicano English being associated with Chicanos.
    However, people who identify with an ethnicity are not necessarily speakers of the related ethnolect, nor are speakers of an ethnolect necessarily members of the related ethnic group.
    For example, \textcite{eckert_where_2008}, in her study of pre-nasal /ae/ raising, found that a Latina who was active in the Fields school ``crowd'' that was dominated by White Anglos followed the raising pattern of that group instead of the pattern one would expect from Latinxs (pp.~37-38).
    % If all speakers of an ethnicity spoke an ethnolect, she would not
    Likewise, \textcite{holliday_multiracial_2019} took individuals with one White and one Black parent and categorized them based on how they self-identified.
    She found that the intonational patterns of her participants varied relatie to these self-identifications (p.~8).
    Had they been grouped into a single broader category such as ``biracial'' or ``mixed race'', these differences would have been lost within a single language variety.
    This sort of incongruence between ethnicity and ethnolect is can also be found in varieties that are often considered different languages, such as Cajun French and Louisiana Creole.

  \section{Cajun or Creole}
    Two major ethnic groups can be found in south Louisiana: Cajuns and Creoles.
    What defined one as a member of these ethnic groups has changed over time, though.
    In the early 20th century, those who were rural and French-speaking were considered Cajun \parencite[p.~198]{smith_influence_1939}.
    Del Sesto \& Gibson (1975) later added to this definition the idea that Cajuns are Catholic, as well \parencite[as cited in][p.~15]{neumann_creole_1985}.
    Today, Cajuns are perhaps most generally understood to be any rural White people in south Louisiana.

    Creoles, on the other hand, have been defined in numerous ways.
    The term was used widely in the New World, initially used to refer to Europeans born there \parencite[Corominas 1967, as cited in][p.~16]{holm_introduction_2000}.
    Eventually, Creole simply meant those of mixed race in south Louisiana, though today they are most generally understood to be Black people from the same area \parencite[Dominguez, 1977, as cited in][p.~11]{neumann_creole_1985}.

  \section{Ethnolanguages}
    The languages associated with these two ethnic groups are Cajun French and Louisiana Creole.
    Naturally, one way to define these languages is by saying that the former is spoken by Cajuns and the latter by Creoles.
    However, this is not the only way; \textcite{klingler_probleme_2005} did so according to three linguistic features: the first person singular subject pronoun being either \lexi{je} (Cajun) or \lexi{mo} (Creole), the perfective being expressed with the auxiliary verb \lexi{avoir} and a past participle (Cajun) or with an uninflected form of a verb (Creole), and the use of \lexi{avoir} for \gloss{to have} (Cajun) or \lexi{gen} for \gloss{to have} (Creole) (pp.~354-355).

    Proceeded by defining these languages according to the ethnic groups who supposedly speak them is unsatisfying as the resulting varieties lack systematicity.
    They may differ in all three of the features that Klingler identified or more.
    In fact, \textcite{klingler_if_2003} found that the majority of those who spoke what he considered to be Creole along linguistic lines were non-Black and so presumably not Creole in identity (p.~108).
    Some of his informants described their speech as ``French'' or ``Creole'' \parencite[p.~128]{klingler_if_2003}, likely because they identified as Cajun instead of Creole.

  \section{White and Black}
    Part of the issue with the difficulty in associating Cajun French and Louisiana Creole with their eponymous ethnicities may come from how these ethnicities have changed in definition to fit into the broader concepts of race in the United States.
    Up until the 20th century, Louisiana was relatively isolated from the rest of the country both culturally and economically \parencite[p.~28]{johnson_louisiana_1976}.
    As such, their concepts of race and ethnicity were not exactly analogous to the typical Black versus White dichotomy that one finds in the US generally.
    As the state became more integrated with the rest of the country, an assimilation occurred, at least in terms of how they treat race.
    Attempting to fit their racial and ethnic categories into the Black and White dichotomy has in effect rewritten the lines that divide groups of people in the state so that those who speak varieties that may be called Creole along linguistic lines may have identified as Creole in the past, but today some identify as Creole and some as Cajun.
    \printbibliography
\end{document}
