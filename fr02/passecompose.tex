\documentclass{article}
  % Packages and package settings
  \usepackage{fontspec}
    \setmainfont{Charis SIL}
  \usepackage{hyperref}
    \hypersetup{colorlinks=true,
                allcolors=blue}
  \usepackage[backend=biber, style=apa]{biblatex}
    \addbibresource{References.bib}
  % \usepackage{graphicx}
  %   \graphicspath{{./figures/}}
  \usepackage[french]{babel}

  % Document information
  \author{Joshua McNeill}
  \date{\today}
  \title{Les auxiliaires pour le passé composé en français louisianais}

  % Custom commands
  \newcommand{\lexi}[1]{\textit{#1}}

\begin{document}
  \maketitle
  \section{Introduction}
    Dans l'état de la Louisiane aux États-Unis, il y avait environ 142.000 francophones en 2010, 3,45\% de la population \parencite{mla_louisiana_nodate}.
    Ce nombre représente une forte diminution des locuteurs du français dans l'État.
    Quarante ans avant, il y en avait environ 570.000, 15,83\% de la population \parencite[p.~159]{us_department_of_commerce_1970_1973}.
    Encore 31 ans avant, en 1939, \textcite{smith_influence_1939} ont deviné que 90\% de la population du sud de la Louisiane parlait français (p.~198).
    C'est-à-dire que le français a été la langue majoritaire de l'État, mais au cours d'un siècle, il est devenu une langue minoritaire.
    Or, cette variété existe toujours avec ses propres caractéristiques.

    L'une des caractéristiques qui distingue cette variété que l'on peut appeler le français louisianais\footnote{Il s'appelle parfois le français cadien, aussi.} est son système pronominal.
    Pour le pronom sujet de la troisième personne du pluriel, en plus d'\lexi{ils}, il y a également \lexi{eux}, \lexi{eux autres} et \lexi{ça} (\citeauthor{brown_pronominal_1988}, \citeyear[pp.~152-153]{brown_pronominal_1988}; \citeauthor{byers_defining_1988}, \citeyear[pp.~88-92]{byers_defining_1988}; \citeauthor{rottet_language_1995}, \citeyear[p.~175]{rottet_language_1995}; \citeauthor{smith_morphosyntactic_1994}, \citeyear[p.~60]{smith_morphosyntactic_1994}).
    Le français lousianais a aussi un pronom sujet qui indique explicitement la deuxième personne du pluriel: \lexi{vous autres} \parencite[p.~175]{rottet_language_1995}.

    Encore un autre trait grammatical que l'on peut exprimer explicitement en français louisianais est l'aspect progressif.
    Cet aspect est exprimé par la construction \lexi{être après} + un verbe à l'infinitif, tel que dans l'Exemple \ref{ex:progressif}, qui veut dire qu'il est en train de parler.
    C'est également possible d'exprimer l'aspect progressif au passé en conjuguant le verbe \lexi{être} à l'imparfait \parencite[p.~102]{papen_structural_1997}.

    \begin{enumerate}
      \item \label{ex:progressif} Il est après parler.
    \end{enumerate}

    Les caractéristiques décrites dessus sont intéressantes à part entière, mais celle dont nous nous occupons ici, c'est le verbe auxiliaire dans le passé composé.
    Selon \textcite{luscher_emplois_1996}, le passé composé peut indique le temps du passé avec soit l'aspect perfectif soit l'aspect accompli (p.~192).
    En français louisianais, le sens est le même, mais la construction diffère.
    En français standard, on emploie d'habitude l'auxiliaire \lexi{avoir} suivi d'un participe du passé.
    On emploie également l'auxiliaire \lexi{être} suivi d'un participe du passé si le verbe signifie un genre de mouvement ou de changement ou si le verbe est réfléchi.
    Ces deux auxiliares sont possible en français louisianais, aussi, mais \lexi{être} n'est pas forcément utilisé par tous les locuteurs.

    % Previous work on variation in the auxiliary verb
      % In Louisiana
      % In other regions
    % Variable and research question
  \section{Méthode}
    % Corpus
    % Data collection and coding
    % Statistics
  \section{Résultats}
    % Overall frequencies
    % Social factors (qualitative)
      % Age
      % 1st language
    % Linguistic factors
      % Reflexive vs verb
  \section{Discussion}
  \printbibliography
\end{document}
