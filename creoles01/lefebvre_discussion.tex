\documentclass{article}
  % Packages and settings
  \usepackage{fontspec}
    \setmainfont{Charis SIL}
  \usepackage{graphicx}
    \graphicspath{{figures/}}

  % Document information
  \title{Overview of Lefebvre (2001)}
  \author{Joshua McNeill}
  \date{14 February 2020}

  % New commands
  \newcommand{\orth}[1]{$\langle$#1$\rangle$}
  \newcommand{\lexi}[1]{\textit{#1}}
  \newcommand{\image}[1]{
    \begin{tabular}[t]{c}
      \\
      \includegraphics[scale=0.7]{#1}
    \end{tabular}
  }

\begin{document}
  \maketitle
  \begin{enumerate}
    \section{Introduction}
      \item Some definitions
      \begin{enumerate}
        \item Relexification: ``[T]he process of vocabulary substitution in which the only information adopted from the target language in the lexical entry is the phonological representation'' (Muysken 1981)
        \item Dialect leveling: ``[T]he reduction of variation between dialects of the same language in situations where these dialects are brought together.'' (p.~372)
      \end{enumerate}
      \item What is the overall purpose of this article?
      \item \image{general_model.jpg}
      \begin{enumerate}
        \item What are the theoretical assumptions? What theoretical framework is Lefebvre working under?
        \item What is ``relabelling''? % the process of assigning a new phonological representation to a copied lexical entry
        \begin{enumerate}
          \item What is [ɸ]? % a null form when there isn't a form available in the lexifier
        \end{enumerate}
      \end{enumerate}
  \newpage
      \item Relexification
      \begin{enumerate}
        \item What drives the process? % According to Muysken, semantics, in that there must be some meaning shared between the two lexical entries
          % Relatedly, how do speakers know the semantics of the form in the lexifier that they don't speak? That's the semantic and pragmatic contexts part
        \item Lefebvre spends some time arguing that creole genesis is incomplete (pp.~375-376), ``crystalized'' L2 acquisition. How might creole genesis fit in with the SLA literature?
        \item \image{sub_variation.jpg}
      \end{enumerate}
      \item Dialect leveling
      \begin{enumerate}
        \item When does leveling start to happen? % When the incipient creole is used between community members (377)
          % Relatedly, how does this compare to Baker's MCS idea?
        \item Lefebvre explains that some features never level out (p.~377). Why might this be?
        \item What do Haitian Creole (HC) and Martinican Creole have to do with leveling? % Similar histories with languages, but leveling process was not identical (378)
      \end{enumerate}
    \section{The Haiti Situation}
      \item Haitian Creole came together between 1680 and 1740 (Singler 1996)
      \begin{enumerate}
        \item Which part of the population was responsible? % Adults, because Haiti was mostly adults
        \item African languages present: Atlantic, Mande, Kwa (including Akan and Gbe), Gur, Nigerian Benue-Congo, Ijoid and Bantu (p.~379)
        \item European languages present: French, but which variety? % Langue d'oïl (Western and Central France)
        \begin{enumerate}
          \item HC \lexi{bay} [bɑj] `to give' from the regional French \lexi{bailler} [bɑje] instead of \lexi{donner} [done] (p.~381)
          % Lefebvre argues that this sort of French origin for creole features doesn't account for much
        \end{enumerate}
      \end{enumerate}
    \section{Features}
      \item 3P pronouns and plural markers
      \begin{enumerate}
        \item \image{plural_fongbe.jpg}
        \item Ewe has \lexi{wò} for both
        \item \image{plural_haitian.jpg}
        \item In what ways could we potentially explain this HC feature's relationship to the languages that were present?
          % Early leveling (Lefebvre's choice)
          % Fongbe speakers used eux for both Fongbe entries (but what abou the semantics?)
          % Leveling BEFORE relexification (with non mutually intelligble languages)
      \end{enumerate}
      \item Reflexives
      \begin{enumerate}
        \item \image{reflex_fongbe.jpg}
        \item Other present languages used \textsc{body}-part reflexives
        \item \image{reflex_haitian.jpg}
          \image{reflex_haitian_more.jpg}
        \item In what ways could we potentially explain this HC feature's relationship to the languages that were present?
        % Lefebvre thinks this is a case of leveling still not occurring even today
        % 9a is relexified to a null form, 9b and 9c to body parts because the semantics were applicable
      \end{enumerate}
  \newpage
      \item Demonstratives
      \begin{enumerate}
        \item \image{demonstratives.jpg}
        \item \image{demo_features.jpg} % she argues that ça, cela, and celui-là are all +-prox, which is false today
        \item In what ways could we potentially explain this HC feature's relationship to the languages that were present?
        % These three semantics match those possible in Fongbe, as well
      \end{enumerate}
    \section{Conclusion}
      \item Did Lefebvre sufficiently support her argument?
  \end{enumerate}
\end{document}
