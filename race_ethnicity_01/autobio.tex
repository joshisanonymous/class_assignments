\documentclass{article}
  % Packages and settings
  \usepackage{fontspec}
    \setmainfont{Charis SIL}
  \usepackage{setspace}

  % Document info
  \author{Joshua McNeill}
  \title{Ethnolinguistic Autobiography}
  \date{\today}

\begin{document}
  \maketitle
  \doublespacing
  My personal ethnic background has never been particularly salient in my life.
  This is especially true as it pertains to language.
  Superficially, my world has always looked generically White middle-class American.
  I believe my parents have always put a premium on fitting into that mold, which was not too difficult to do in the fairly homogenous part of South Jersey where I grew up.
  Of course, the reality was not as generic, but that only became obvious to me later in life.

  I grew up in a seaside town in New Jersey that was not particularly diverse.
  My neighborhood in particular, which was really the most relevant to me before entering school, was entirely White, Protestant, and squarely middle-class.
  There were a number of other boys my age in who lived close enough that we could play together regularly.
  We typically played popular American sports, and at the time, we all had very similar speech patterns.
  Without any obvious contrasts, it never occurred to me that there might be something different about my background from theirs.

  My family, at least to me, seemed equally uninteresting in terms of ethnicity.
  However, one grandparent on each side of the family was from the South.
  My grandfather grew up in Tennessee and Alabama, and this was very clear in his speech.
  He's been gone some time now, but to this day, I can still hear him saying, ``I'll tell you what,'' with a monophthongized \emph{I'll} [ɑːl] and a distinctive intonation.
  This idea of being southern or even that these sort of speech patterns were southern never even occurred to me, though.
  In part, this is because my grandmother, who spent her whole life in South Jersey, seemed to speak just like her husband.
  It turns out that, at one time, many of the linguistic features common in my area of Jersey resemble those of the South.
  For me, as a young kid, southern-sounding speech was indexed more as ``old person English.''

  The other side, my father's side, was similar in a lot of ways.
  My grandmother was a distinctly Cajun woman from Louisiana.
  She spent most of her life there, and although I didn't find out until much later, her first and primary language was French until she left the state.
  Likewise, my oldest aunt didn't speak English until she started school, which explains why she always sounded much more like my grandmother in English than my other aunts.
  While Cajun English is most definitely distinct from other southern varieties, to me, this was just an extension of old person English.
  As for my grandfather, I don't remember what he sounded like, but he was the local New Jerseyian of the pair.

  While my parents naturally picked up some of the features of the way my grandparents spoke, it was not in particularly noticeable ways.
  My father, in particular, wanted to distance himself from his impoverished Cajun background, and so emphasized his Scottish ancestry from his father's side whenever ethnic identity came up.
  My mother, while not having a negative relationship with her upbringing, did much the same.

  I spent a lot of time with my cousins when I was younger, and we would pick up lexical features of African-American English that we heard from hip-hop groups coming out of Philadelphia.
  It cool to use words like \emph{bad} `good', though I didn't associate these with African-American identity \emph{per se}, perhaps because they were distant group in my White region of Jersey.

  I left New Jersey in my early 20s and have such lived in various parts of the country and even Canada.
  These experiences have made personal identity and ethnicity much more salient issues for me.
  It became fun to point out to people that I grew up saying [wʊ.ɾəɹ] \emph{water} and referring to subs as \emph{hoagies}.
  These simple things didn't feel like they said anything about who I was until I wasn't where I came from anymore.

  When I moved to New Orleans, it was in part because I was interested in my family's background, which was mostly hidden from me when I was younger.
  It was then that I learned that my grandmother spoke French, and then that I started to embrace the southern aspects of my identity.
  The southern aspects had always been there, but they were so subtle that I didn't notice them myself.
  For instance, a friend of mine once asked me why I suddenly sounded southern, and I then realized that in certain contexts, I channeled my Alabamaian grandfather.
  So, I took it upon myself to learn Louisiana French and immerse myself in all things Louisiana, which has become a big part of who I am and part of my linguistic repertoire.
  There is always a balancing act, though: I am not Cajun, even if I know more about Cajun culture than many of those who definitively are Cajun, and I'm not southern even though I've spent 6 years in South and grew up surrounded by southern grandparents.
  I can exploit parts of my southern linguistic repertoire while in the south, particularly French, though only to subtle degrees for fear of seeming inauthentic.
  I certainly can't exploit those parts with my friends in New Jersey, although it's okay with my family.
  Ironically, aspects of my speech that would place me as being from South Jersey now feel more performative, as well, as I bought into the educational system's concept of correct speech more than most from the area.
  In a way, my ethnicity feels like something that I draw on at times in my language, but not something that I necessarily live.
\end{document}
