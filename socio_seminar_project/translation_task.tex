The idea behind this translation task is to elicit tokens of the third person plural subject pronoun either with \emph{être}/\emph{avoir} or with any other verb. Furthermore, potential referents are divided up according to animacy (animate vs inanimate), specificity (generic vs specific), and distance (referent in the T-unit vs not in the T-unit). The phrases themselves should be randomly ordered when presented to participants.

\begin{enumerate}
    \subsection{\emph{être}/\emph{avoir}}
        \item (Specific-Animate-Close) My kids, they have kids, too.
        \item (Generic-Animate-Close) People, they're strange sometimes.

        \item (Specific-Inanimate-Close) Those cracklins, they're good.
        \item (Generic-Inanimate-Close) Things, they are what they are.

        \item (Specific-Animate-Distant) My parents have a nice house. They have figs in their yard.
        \item (Generic-Animate-Distant) They closed the school.

        \item (Specific-Inanimate-Distant) He makes accordions. They're real nice.
        \item (Generic-Inanimate-Distant) Things happen. They're just unavoidable.
    \subsection{Regular Verbs}
        \item (Specific-Animate-Close) Those doctors, they live in Lafayette.
        \item (Generic-Animate-Close) People, they like to dance.

        \item (Specific-Inanimate-Close) Those roads, they run from here to Lafayette.
        \item (Generic-Inanimate-Close) Things, they seem fishy around there.

        \item (Specific-Animate-Distant) My neighbors are annoying. They park their cars in front of my house.
        \item (Generic-Animate-Distant) They talk about how nice that store is all the time.

        \item (Specific-Inanimate-Distant) Card games are fun. They give people a reason to get together.
        \item (Generic-Inanimate-Distant) Things would be good to have. They cost a lot, though.
\end{enumerate}
