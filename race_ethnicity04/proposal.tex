\documentclass{article}
  % Packages and settings
  \usepackage{fontspec}
    \setmainfont{Charis SIL}
  \usepackage{setspace}
  \usepackage[style=apa, backend=biber]{biblatex}
    \addbibresource{References.bib}
  % \usepackage{graphicx}
  %   \graphicspath{{figures/}}

  % Document information
  \title{Comparison of the habitual present between Black, White, and Creole speakers of Louisiana Creole (proposal)}
  \author{Joshua McNeill}
  \date{\today}

  % New commands
  \newcommand{\orth}[1]{$\langle$#1$\rangle$}
  \newcommand{\lexi}[1]{\textit{#1}}
  \newcommand{\gloss}[1]{`#1'}
  \newcommand{\image}[1]{
    \begin{tabular}[t]{c}
      \\
      \includegraphics[scale=0.7]{#1}
    \end{tabular}
  }

\begin{document}
  \maketitle
  \doublespacing
  \section{Introduction}
    The parish\footnote{Parishes in Louisiana are equivalent to counties in other US states.} of Pointe Coupee, Louisiana is situated along the Mississippi River northwest of Baton Rouge.
    It is a rural area whose population in 1990 was 22,540 and in 2000 was 22,763 \parencite{us_census_bureau_census_2001}.
    That is to say, it is not an area that expands or contracts much, which has been true over much of its history.
    In the 19th century, \textcite{klingler_if_2003} claims that it was so insular that marriage between cousins was not exceptional (p.~109).
    Although much of South Louisiana was relatively isolated until up through the first half of the 20th century (\citeauthor{gold_french_1979}, \citeyear{gold_french_1979}; \citeauthor{johnson_louisiana_1976}, \citeyear{johnson_louisiana_1976}, p.~28), Pointe Coupee perhaps stands out even from other parishes.

    In Pointe Coupee, a variety of Louisiana Creole is spoken.
    Another major variety is spoken in St Martin Parish \parencite{neumann_creole_1985}, though the one spoken in Pointe Coupee stands out as St Martin is much less isolated than Pointe Coupee, both geographically and culturally as it is part of the Lafayette metropolitan area where Louisiana French is also spoken.
    Louisiana Creole itself is a French lexifier creole that has many of the features commonly associated with other French lexifier creoles such as preverbal markers for tense, aspect and mood, post-nominal definite articles, and copula absence \parencite[p.~63]{klingler_if_2003}.
    It also has much the same origin, being born out of French plantation society and specifically slavery.

    There are three racial distinctions that the people of Pointe Coupee make: White, Black, and Creole of color, meaning those of mixed race \parencite[p.~xxv]{klingler_if_2003}.
    Despite this tertiary division -- as opposed to the typical binary division in most of the United States -- people of color in Pointe Coupee have experienced much of the same racial issues as Blacks generally in other parts of the country.
    For instance, there was not a high school that people of color could go to in Pointe Coupee until 1950, before which there was sometimes violent opposition to such initiatives \parencite[p.~122]{klingler_if_2003}.
    A surprising fact, given Louisiana Creole's roots in slavery, is that the variety spoken in Pointe Coupee is spoken by more White people than Black or Creole people, which \textcite{klingler_if_2003} believes to be the result of Americans imported English-speaking slaves in the 19th century and so providing slaves with more exposure to English than slave owners even (p.~108).

    One variable feature in Louisiana Creole is the expression of the habitual present, which may be done using an uninflected bare very or an inflected verb, in both cases without tense, mood, or aspect markers \parencite[pp.~237]{klingler_if_2003}.
    For instance, the verb \lexi{parle} [paɾle] \gloss{to speak} may express the habitual present either as in

    \begin{tabular}{l r l}
      1. & Uninflected & Li parle \gloss{He speaks} \\
      2. & Inflected   & Li parl \gloss{He speaks}
    \end{tabular}

    For stative verbs, the so-called long form (i.e., inflected with the \emph{-e} suffix) can also express the perfective past, so that (1) may also mean \gloss{he spoke} \textcite[p.~252]{klingler_if_2003}.

    Though he provides no quantitative analysis, \textcite{klingler_if_2003} claims that the use of uninflected versus inflected forms of verbs to express the habitual present is socially constrained: Whites prefer the inflected version and Blacks the uninflected (p.~117).
    As it happens, the inflected forms are also often identical to their counterparts in Louisiana French (e.g., \lexi{il parle} [paɾl] \gloss{he speaks}), which is considered more prestigious than Louisiana Creole at least among Creole speakers in St Martin Parish \parencite[pp.~23-25]{neumann_creole_1985}.
    My goal with this study is to carry out a quantitative analysis of Klingler's data to confirm or refute Klinger's claim that this variation is racially constrained.
    As such, my first research question is the following:
    \begin{enumerate}
      \item[Q1] Do Black, White, and Creole speakers of Louisiana Creole differ in their preference for uninflected versus inflected verbs when expressing the habitual present?
    \end{enumerate}
    As this may have implications for racial identities in the parish, I will also ask the following qualitative research question:
    \begin{enumerate}
      \item[Q2] What do the differences or similarities between the forms chosen for the habitual present in Louisiana Creole mean for how different races in Pointe Coupee Parish identify themselves and their relations to each other?
    \end{enumerate}
  \section{Methodology}
    \subsection{Corpus}
      The corpus for this study comes from \citeauthor{klingler_if_2003}'s (\citeyear{klingler_if_2003}) study.
      In his book, he provides lengthy extracts from the interviews he conducted during his own research, along with both descriptions and demographic details about the interviewees.
      Out of the Interviews provided, two are with White speakers, two are with Creole speakers, and the rest are with Black speakers.\footnote{It is important to note here that Klingler does not make it clear whether he classified his speakers or whether they self-identified, though I assume the latter -- or that the classifications are at least accurate -- as it is hard to imagine how he would have identified any of the speakers as Creole instead of Black without asking.}
      Each involves roughly 300 utterances by the interviewees.
      As there is more data from Black speakers than those of other races, only two interviews with Black speakers will be chosen to keep the data balanced.
      All of the interviews were carried out in the early 1990s.
      Though this is not ideal for what the situation may be today, this is some of the only data available on these speakers, and it is quite possible that no speakers are left as an interviewee being 52 was considered ``younger'' \parencite[p.~381]{klingler_if_2003}.

    \subsection{Data collection and coding}
      Tokens of verbs will be extracted from the corpus and coded manually.
      Some verbs in Louisiana Creole have one invariant form, such as \lexi{di} \gloss{to say} \parencite[p.~247]{klingler_if_2003}.
      These verbs will be disregarded in this analysis.
      Those that have two forms are spelled as such.\footnote{Klingler uses his own orthography based on the Haitian Creole writing system.}
      The two most regular verbs are those that end in \emph{-e}, as in \lexi{parle} in (1-2), and those that end in \emph{-n}, such as \lexi{konprann} [kɔ̃pɾɑ̃n] \gloss{to understand}.
      In both cases, the inflected forms simply leave off the final segment \parencite[p.~244-246]{klingler_if_2003}.
      Since Klingler's writing system is quasi-phonetic, coding will consist of identifying whether he transcribed these verbs with or without the final segment.

      As the goal is to find out what exactly is influencing the realizations of these verbs as uninflected or inflected, each token will be coded for the gender, age, and place of origin of the speaker, in addition to race.
      It is also possible that verb types play a role, as stative and non-stative verbs have indiosyncracies in other aspects of the grammar \parencite[pp.~252-253]{klingler_if_2003}.
      As such, tokens will also be coded for whether they are stative or non-stative.

    \subsection{Statistical analysis}
      For the quantitative analysis, binomial logistic regression will be used with the verb form as the response variable, uninflected or inflected being the two possible values.
      Each of the factors that the tokens were coded for will be considered in the analysis.
    \printbibliography
\end{document}
