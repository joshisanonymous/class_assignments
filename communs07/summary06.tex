\documentclass{article}
  % Packages and settings
  \usepackage{fontspec}
    \setmainfont{Charis SIL}
  \usepackage[margin=1.0in]{geometry}
  \usepackage[style=apa, backend=biber]{biblatex}
    \addbibresource{References.bib}

  % Document information
  \title{Summary of Wright (2007) Chapter 2}
  \author{Joshua McNeill}
  \date{12 November 2020}

  % New commands
  \newcommand{\orth}[1]{$\langle$#1$\rangle$}
  \newcommand{\lexi}[1]{\textit{#1}}
  \newcommand{\gloss}[1]{`#1'}

\begin{document}
  \maketitle
  \fullcitebib{wright_understanding_2015}
  \section{Introduction}
    Exploitation as a general concept central to class conflict is systematically ignored in the Weberian tradition.
    This chapter's objective then is twofold:
    \begin{itemize}
      \item To understand Weber's concept of class and how it contrasts with Marx's concept and exploitation.
      \item To defend the importance of exploitation in sociological work.
    \end{itemize}
  \section{Weber's class and Convergences with Marx's}
    Weber's conceptualization of class:
    \begin{itemize}
      \item Class locates individuals only within the economic sphere of social interaction, not the communal nor the political spheres.
      \item Class, then, is not a group as there is no shared identity.
      \item Additionally, class is the rationalization of social relations relative to one's material conditions
    \end{itemize}
    This conceptualization aligns with Marx's in several ways.
    Classes are relational, defined in opposition to each other.
    Having or not having property is central to one's class.
    Neither view classes as organized social actors, but one's relation to materials influences their behavior in both conceptualizations, and the preconditions to organization do exist in capitalist societies.
    Finally, both view class as in competition with status groups.
  \section{Differences between Weber and Marx}
    In general, Weber saw class as a determinant of life chances where Marx saw this as well as its role in exploitation.
    For Weber, class involves rationalized decisions in the market, but for Marx, exploitation is also involved so that slaves, for instance, form a class but for Weber a status group.
    This has several ramifications:
    \begin{itemize}
      \item The research questions we ask and how we approach those questions are shaped by which framework we use.
      \item One's agenda is also shaped by the framework: Weber's work is oriented towards efficiency for capitalists and Marx towards the plight of workers.
    \end{itemize}
  \section{Discussion}
    \begin{enumerate}
      \item If it's possible to be sympathetic towards those being exploited without having exploitation as a formal concept in class framework, then how much of an impact does the framework really have on one's agenda?
      \item How would the concept of exploitation change our interpretation of class-related phenomena in sociolinguistics such as hypercorrection or covert prestige?
    \end{enumerate}
\end{document}
