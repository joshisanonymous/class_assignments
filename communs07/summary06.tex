\documentclass{article}
  % Packages and settings
  \usepackage{fontspec}
    \setmainfont{Charis SIL}
  \usepackage[margin=1.0in]{geometry}
  \usepackage[style=apa, backend=biber]{biblatex}
    \addbibresource{References.bib}

  % Document information
  \title{Summary of Podesva (2007)}
  \author{Joshua McNeill}
  \date{29 October 2020}

  % New commands
  \newcommand{\orth}[1]{$\langle$#1$\rangle$}
  \newcommand{\lexi}[1]{\textit{#1}}
  \newcommand{\gloss}[1]{`#1'}

\begin{document}
  \maketitle
  \fullcitebib{podesva_phonation_2007}
  \section{Introduction}
    \noindent{}\citeauthor{podesva_phonation_2007}'s (\citeyear{podesva_phonation_2007}) objective with this article is to show that variable phonation types, particularly falsetto, can be used to construct personae.
    \begin{itemize}
      \item Falsetto is defined as lengthening and adducting of the vocal folds, leading to higher F0 values.
      \item To do this, he analyzes the speech of Heath, a gay medical student, in various situations.
    \end{itemize}
    Work on phonation, especially in sociolinguistics, has been scarce, but there have been some notable examples.
    \begin{itemize}
      \item Most work has used impressionistic analyses by trained phoneticians and focused on associations with communities and groups.
    \end{itemize}
  \section{Falsetto}
    \noindent{}Data was collected by having Heath record conversations in his daily life.
    \begin{itemize}
      \item Three of the recorded situations were analyzed: 1) a barbecue with four friends, 2) a phone conversation between Heath and his father, and 3) a meeting with a patient.
      \item Sentences containing falsetto, judged impressionistically, were coded as such.
    \end{itemize}
    Out of the three situations, Heath used falsetto over twice as much at the barbecue (9.07\%).
    \begin{itemize}
      \item In particular, the durations of the BBQ falsettos are also longer than in other situations.
      \item Relatedly, creaky voice durations were also longer during the BBQ.
    \end{itemize}
  \section{Social Meaning}
    Podesva argues that falsetto is used to construct a flamboyant diva persona.
    \begin{itemize}
      \item This is something of an artifact of using falsetto to carry out various discourse functions, importantly sharing the notion of ``expressiveness''.
      \item He argues that this persona is an attitude that Heath performs as opposed to performing a gay identity.
    \end{itemize}
  \section{Discussion}
    \begin{enumerate}
      \item Is Podesva's explanation for why the term \lexi{diva} is the most appropriate for describing Heath's style convincing?
      \item Does Heath have to be aware of the work his falsetto is doing in order to use it to do that work?
      \item How should we understand identity in light of Podesva's discussion at the end of the Gay Identity section?
    \end{enumerate}
\end{document}
