\documentclass{article}
  % Packages and package settings
  \usepackage{fontspec}
    \setmainfont{Charis SIL}
  \usepackage{hyperref}
    \hypersetup{colorlinks=true,
                allcolors=blue}

  % Title info
  \author{Joshua McNeill}
  \title{Homework 6}
  \date{\today}

  % Custom commands
  \newcommand{\lexi}[1]{\textit{#1}}

\begin{document}
  \maketitle
  \begin{enumerate}
    \item
    \begin{itemize}
      \item {[}wɛn̠ ʃi təɹnd̚ bæk ðə mæn wəz̥ ɹəɡɑɹdɪŋ əɹ əbʌv ɪz̥ sliv wɪtʃ i hæd̚ pɹɛst̚ tə ðə kʰʌt ɔn ɪz tʃik xi æd̚ pʰʌt ɔn ə nɪɾəd̚ hæt̚ wɪtʃ wəz soʊ pʰɔɹli meɪd̚ xi maɪt̚ wɛl æv pʰʌɾ ɪt̚ tʰəɡɛðɹ̩ ɪmsɛlf fɹəm skɑɹlət̚ skɹæps ən puld ɪt̚ tʰu ɪz aɪbɹaʊz wɪtʃ wəɹ θɪk wɪð mʌd̚ ɛn ɔlmoʊst̚ əpskjʌɹd̚ ɪz aɪz xi sɛd̚ θæŋks ə lɪɾl̩ kɹ̩t̚li əɡɛn wɪð ðæt̚ flæʔn̩ɪŋ əv ðə vaʊlz̥ ðæt̚ mɑɹkt ɪm aʊt̚ æz ə kʌntʃɹi mæn ə fɑɹməɹ ðɛn ʃi θɔt̚ ɛn wɪðaʊt̚ ək̚sɛp̚tɪŋ ðə ɡɹæɾɪtʰud soʊ ɡɹʌdʒɪŋli ɡɪvn̩ ʃi sɛd̚ dʒɛʃtʃəɹɪŋ tu ðə ɛɡ̚zɔstʰəd̚ ʃip̚ ɪz ɪt̚ ɡoʊwɪn ɾə bi ɑl ɹaɪt̚ ɪt̚ maʊθt̚ æt̚ ðə ̝ɛɹ ɛn ɹɔld̚ ɪt̚s aɪz əɡɛn xi ʃɹʌɡd ʃʊd d̪ɪŋk soʊ wʌn əv jəɹz hɑ noʊ nɑt̚ maɪ flɑk ðə aɪdiə ɛvɪdɛnt̚li ʃtʃɹʌk sʌm kʰɔɹd əv sloʊ hjuməɹ ɪn ɪm ɛn i bəɡæn ɾə tʃʌkl̩]
      \item Several aspects of my reading of the passage stand out in terms of pronunciation.
      The plural allmorph in \lexi{vowels} [vaʊlz̥] is difficult to transcribe.
      I definitely devoice the [z], but sometimes there is still some voicing at the very beginning of the segment.
      In fact, the same thing seems to happen with the [z] in \lexi{was} and \lexi{his} at the beginning of the passage.
      Also, I seem to regularly produce /h/ as [x] before [i] in words like \lexi{he}.

      Additionally, in \lexi{country}, instead of simply retracting the [t], I convert this segment into the affricate [tʃ].
      This occurs consistently before [ɹ] in my speech.
      The fricative that I have right before the affricate in \lexi{gesturing} is assimilated to the fricative of the affricate: [dʒɛʃtʃəɹɪŋ].
      It seems the typical American English fricative in that position is [s], but it is quite awkward for me to produce that sound in that position.
      Both the affrication process in \lexi{country} and the assimilation process in \lexi{gesturing} occur at the beginning of \lexi{struck} so that it sounds like [ʃtʃɹʌk].

      It is not clear to me why, but the vowel I produce in \lexi{to} is sometimes [u] and sometimes [ə].
      Perhaps this has something to do with where the most stress in the clause is, but I am not sure.
      At the end of the passage, the sequence \lexi{began to} leads to the [t] in \lexi{to} being pronounced as [ɾ], presumably because it is intervocalic and unstressed, even though it is a separate word.
      This doesn't happen in any of the other tokens for me.
      The [ɛ] in \lexi{air} seems to be lowered for me so that it almost sounds like [æ].
      This might only be occurring before [ɹ], as the sentence \lexi{It mouthed at the air, and rolled its eyes again} ends with the [ɛ] in \lexi{again} pronounced in a very neutral way.

      Finally, in the series of words \lexi{should think}, I am definitely creating a dental articulation at the beginning of \lexi{think}, but it does not seem to be a fricative.
      It seems to be a stop, like a dental [d].
      I imagine this is because of the preceding [d] in \lexi{should}.
    \end{itemize}
  \end{enumerate}
\end{document}
