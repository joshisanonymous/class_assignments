\documentclass{article}
  % Packages and settings
  \usepackage{fontspec}
    \setmainfont{Charis SIL}
  \usepackage{setspace}
  \usepackage[style=apa, backend=biber]{biblatex}
    \addbibresource{References.bib}
  % \usepackage{graphicx}
  %   \graphicspath{{figures/}}

  % Document information
  \title{Summary of Myers-Scotton (2001) and Sánchez (2006)}
  \author{Joshua McNeill}
  \date{\today}

  % New commands
  \newcommand{\orth}[1]{$\langle$#1$\rangle$}
  \newcommand{\lexi}[1]{\textit{#1}}
  \newcommand{\gloss}[1]{`#1'}
  \newcommand{\titl}[1]{``#1''}

\begin{document}
  \maketitle
  \onehalfspacing
  \section{Myers-Scotton (2001)}
    In \citeauthor{myers-scotton_implications_2001}'s (\citeyear{myers-scotton_implications_2001}) study, entitled \titl{Implications of abstract grammatical structure: Two targets in creole formation}, she proposed to categorize different levels of grammar and morphemes in order to explain why some aspects of creoles are derived from substrates and why others are derived from superstrates.
    The two broad levels of grammar that she distinguishes are the grammatical level and the lexical level.
    She then distinguishes between four types of morphemes: content morphemes and three types of system morphemes.
    Some hypotheses about how creoles form are made that are motivated by these divisions, but the work of systematically testing these hypotheses is left to other researchers.

    Myers-Scotton's first division into a lexical level and a grammatical level is used to conclude that substrate languages constitute the grammatical level in creole formation.
    She claims that the grammatical level remains intact for much longer than the lexical level because it is not as salient, citing second language learners as an example.
    Ultimately, this means that substrate languages form the frame into which content from a lexifier can be inserted during creole formation.
    The morphemes that are then inserted are based on their accessibility hierarchy in language production, with those that have more semantic content being inserted earlier during production and so being more salient and accessible.

    Five hypotheses stem from this framework, then.
    Hypothesis 1 states that substrates should be the source for the composite Matrix Language, which is effectively the same as the grammatical level.
    Hypothesis 2 states that content morphemes from both the substrate and superstrate may be used in the creole, but those from the superstrate are more likely.
    Hypothesis 3 states that superstrate content morphemes can become function morphemes in the creole.
    Hypothesis 4 would claim that early system morphemes, those that are function morphemes but still have semantic content, can be taken from the superstrate but only if combined with the head of the constituent they are in.
    Finally, hypothesis 5 states that late system morphemes, those that are function morphemes with little to no semantic content, will never be taken from a superstrate language.

    Myers-Scotton offers these hypotheses for other researchers to test rigorously.
    She does provide examples for each, but makes it clear that her examples are not sufficient evidence.
    Ultimately, what she offers is a refinement of the idea that creoles take content morphemes from lexifiers but not function morphemes.

  \section{Sánchez (2006)}
    In \citeauthor{lefebvre_bilingual_2006}'s (\citeyear{lefebvre_bilingual_2006}) study \titl{Bilingual grammars and creoles: Similarities between functional convergence and morphological elaboration}, she aimed to draw together the disiplines of bilingualism research and creole studies by showing the similarities between functional convergence in the former and morphological elaboration in the latter.
    She defines functional convergence as a phenomenon that occurs when bilinguals use functional features from their L1 that are absent or unacquired in their L2 by combining these functional features with morphemes from the L2.
    She demonstrates this phenomenon using examples from Quechua-Spanish bilinguals.
    She defines morphological elaboration, on the other hands, as the process by which speakers of creoles rely less and less on lexical morphemes and context to express grammatical features and more so on grammatical morphemes.
    Hawaiian Creole English and Singapore Colloquial English are used to demonstrate this phenomenon.

    Sánchez analyzes two functional features in the speech of Quechua-Spanish speakers: evidentiality and the imminent future.
    Evidentiality refers to whether a speaker has firsthand knowledge that what they are saying is true.
    In Quechua, this is expressed using a past tense suffix, but in Spanish, this is not expressed grammatically at all but must instead be expressed through discourse.
    In her previous, Sánchez had Quechua-Spanish bilinguals relate a story in both Quechua and then in Spanish to elicit the evidential mood.
    These speakers regularly used the pluperfect in Spanish to express that something was secondhand information.
    This pattern had also been attested in several independent studies.

    Likewise, Quechua-Spanish speakers express the imminent future in Spanish using functional morphemes, something which is not normally done in Spanish but is in Quechua.
    Specifically, she reports speakers using the progressive tense with the verb \lexi{querer} \gloss{to want} followed by another verb in the infinitive.
    Both of these examples are described as being similar to what is found in creoles but described instead as morphological elaboration.

    Sánchez's examples of morphological elaboration in creoles come from other researchers, namely Siegel and Bao, so she goes less in depth here.
    The first example comes from Hawaiian Creole English in which \lexi{stay} is used as a progressive marker.
    She notes that the progressive was only attested early on among Portuguese speakers in the form of reduplication and the adverbial \lexi{all time}.
    Portuguese also happens to have an explicit progressive, and so she concludes that this may have been the source of \lexi{stay} in the eventual creole.

    The other example from creoles is in Singapore Colloquial English which has \lexi{already} as a completive aspect marker.
    This is thought to come from Hokkien \lexi{liáu} as speakers of Singapore English typically also speak Hokkien.

    Sánchez admits that there limitations to being able to draw firm conclusions about creoles in the same way one can with bilingual speech, but she suggests that these two disciplines can inform each other in any case.
    Her comparison of functional convergence and morphological elaboration is just one example.
    \printbibliography
\end{document}
