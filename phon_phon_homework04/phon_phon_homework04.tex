\documentclass{article}
  % Packages and package settings
  \usepackage{fontspec}
    \setmainfont{Charis SIL}
  \usepackage{hyperref}
    \hypersetup{colorlinks=true,
                allcolors=blue}
  \usepackage{caption}
  \usepackage{graphicx}
    \graphicspath{{./figures/}}

  % Title info
  \author{Joshua McNeill}
  \title{Homework 4}
  \date{\today}

  % Custom commands
  \newcommand{\lexi}[1]{\textit{#1}}
  \newcommand{\specfig}[2]{
    \begin{minipage}[t]{\linewidth}
      \captionof{figure}{#1}
      \label{fig:#2}
      \centering
      \includegraphics[scale=0.5]{#2.jpg}
    \end{minipage}
  }

\begin{document}
  \maketitle
  \begin{enumerate}
    \item do later
    \item
    \specfig{[i] in \lexi{he}}{ispectrum}
    The spectral peaks of the first three formants of [i] in Figure \ref{fig:ispectrum} are 338Hz (F1), 2993Hz (F2), and 3587Hz (F3).

    \specfig{[e] in \lexi{may}}{espectrum}
    The spectral peaks of the first three formants of [e] in Figure \ref{fig:espectrum} are 622Hz (F1), 2512Hz (F2), and 3194Hz (F3).

    \specfig{[æ] in \lexi{map}}{aespectrum}
    The spectral peaks of the first three formants of [æ] in Figure \ref{fig:aespectrum} are 1020Hz (F1), 2002Hz (F2), and 3194Hz (F3).

    \specfig{[ɔ] in \lexi{Bobby}}{ospectrum}
    The spectral peaks of the first three formants of [ɔ] in Figure \ref{fig:ospectrum} are 590Hz (F1), 1161Hz (F2), and 2863Hz (F3).
  \end{enumerate}
\end{document}
