\documentclass{article}
  % Packages and package settings
  \usepackage{fontspec}
    \setmainfont{Charis SIL}
  \usepackage{hyperref}
    \hypersetup{colorlinks=true,
                allcolors=blue}
  \usepackage{phonrule}

  % Title info
  \author{Joshua McNeill}
  \title{Homework 9}
  \date{\today}

  % Custom commands
  \newcommand{\rulealign}[1]{
    \begin{tabular}[t]{l}
      \\
      #1
    \end{tabular}
  }

\begin{document}
  \maketitle
  \begin{enumerate}
    \item
    \begin{enumerate}
      \item Alveolar stops are deleted when preceding bilabial consonants.
      \item Alveolar nasals assimilate to the place of articulation of the following consonant.
      \item \rulealign{
        \phonr{\phonfeat[l]{-continuant \\ -sonorant \\ coronal \\ +anterior}}{ø}{\phonfeat[l]{+labial \\ -strident}}
      }
      \item \rulealign{
        \phonr{\phonfeat[l]{nasal \\ coronal}}{\phonfeat[l]{$\alpha$labial \\ $\beta$coronal \\ $\gamma$anterior \\ $\delta$dorsal \\ $\epsilon$back \\ $\phi$low}}{\phonfeat[l]{-syllabic \\ $\alpha$labial \\ $\beta$coronal \\ $\gamma$anterior \\ $\delta$dorsal \\ $\epsilon$back \\ $\phi$low}}
        \vspace{2cm}
      }
      \item
      \begin{tabular}[t]{l l l}
        UR           & ɡɹændmʌðɹ̩ & boldkʰal \\
        Deletion     & ɡɹænmʌðɹ̩  & --- \\
        Assimilation & ɡɹæmmʌðɹ̩  & --- \\
        SR           & ɡɹæmmʌðɹ̩  & boldkʰal
      \end{tabular}
      Deletion applies first so that the /d/ in \emph{grandmother} is deleted, though the rule does not apply to the [d] in \emph{bold call} because there is no bilabial sound at the beginning of \emph{call}. Likewise, assimilation applies to \emph{grandmother} since there is now a sequence of /nm/, but not for \emph{bold call}. This yields the correct surface representations.
      \item Rule ordering is essential for \emph{grandmother} but not for \emph{bold call}. If the order was reverse, assimilation would not apply to \emph{grandmother} because there would still be a [d] between the two /n/ at the end of \emph{grand}, but deletion would still apply, yielding [ɡɹænmʌðɹ̩].
    \end{enumerate}
    \item There are two rules that apply:
    \begin{enumerate}
      \item (voicing) \rulealign{
        \phonl{\phonfeat[l]{-sonorant \\ +continuant \\ -voice}}{\phonfeat{+voice}}{\phonfeat{nasal}}
      }
      \item (backing) \rulealign{
        \phonr{\phonfeat[l]{-sonorant \\ +continuant \\ -voice \\ +anterior}}{\phonfeat{-anterior}}{\phonfeat[l]{+syllabic \\ -low \\ -back}}
      }
    \end{enumerate}
    The rules can apply in any order. The reason one would never end up with [ʒ] in words like \emph{umziktʃʌm} (i.e., [um\textbf{ʒ}iktʃʌm]) is because the voicing rule is restricted to alveolar fricatives, otherwise this incorrect surface representation would be possible.
  \end{enumerate}
\end{document}
