\documentclass{article}
  % Packages and settings
  \usepackage{fontspec}
    \setmainfont{Charis SIL}
  \usepackage{setspace}
  \usepackage[style=apa, backend=biber]{biblatex}
    \addbibresource{References.bib}
  % \usepackage{graphicx}
  %   \graphicspath{{figures/}}

  % Document information
  \title{Summary of Heine and Kuteva (2003)}
  \author{Joshua McNeill}
  \date{\today}

  % New commands
  \newcommand{\orth}[1]{$\langle$#1$\rangle$}
  \newcommand{\lexi}[1]{\textit{#1}}
  \newcommand{\gloss}[1]{`#1'}
  \newcommand{\image}[1]{
    \begin{tabular}[t]{c}
      \\
      \includegraphics[scale=0.7]{#1}
    \end{tabular}
  }

\begin{document}
  \maketitle
  \onehalfspacing
  \citeauthor{heine_contact-induced_2003}'s (\citeyear{heine_contact-induced_2003}) article ``On contact-induced grammaticalization'' addresses the question of whether language changes are the result of grammaticalization or language contact.
  Grammaticalization would involve language internal processes only and so could explain language changes even in the absence of contact.
  Obviously, contact is explicitly a language external process in which one language may change by borrowing from another language or immitating some feature in another language.
  One can argue that one of these processes or the other is at play in any given situation, but Heine and Kuteva take the position that both are actually at work in most cases and aim to prove that using examples from numerous contact situations.

  While there are various types of borrowing that one could examine, Heine and Kuteva choose to look strictly at borrowing meanings, excluding the borrowing of forms.
  As their proposal is essentially a sort of cross-linguistic analogy process, they propose referring to the languages involves as the model language $M$ that provides the concept to be borrowed and the replica language $R$ that does the borrowing.
  For Heine and Kuteva, the question is ultimately what exactly is being analogized.

  The first possibility they simply call grammaticalization in a language contact situation.
  Here, spesakers of $R$ simply recognize a grammatical concept in language $M$ and then proceed through their own language internal grammaticalization process to achieve the same result.
  While language contact is involved, it is really universal cognitive mechanisms that create the change, whereas the contact simply triggered the beginning of that process.

  The second possiblity is what Heine and Kuteva call replica grammaticalization.
  In this case, the process is much the same as in the process just described, except that speakers of language $R$ do not simply recognize a grammatical concept from in language $M$ but also attempt to recreate the grammaticalization process that they believe to have occurred in language $M$.
  In this case, they would not be relying on universal cognitive mechanisms per se but imitating the way they believe those mechanisms worked in another language.

  The third possibility is referred to as polysemy copying.
  In these cases, speakers of language $R$ do not attempt to reconstruct the entire internal grammaticalization that they believe to have taken place in language $M$ but only the beginning and end points.
  Essentially, the argument would be that there is no reason to reinvent the wheel when one can simply recognize a round thing being used and then find some round thing of one's own and use it as a wheel also.

  Heine and Kuteva, in a way, muddy the waters a bit between those who believe purely language internal grammaticalization explains language changes and those who believe pure borrowing explains language changes as it adds several other possibilities, but what they conclude is essentially that these several other possibilities are actually the only possibilities.
  They do not seem concerned choosing one over the other but instead acknowledge that each explains different cases.
  They ultimately admit that their article was more about raising questions than putting to rest an old debate.
  \printbibliography
\end{document}
