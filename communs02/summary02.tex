\documentclass{article}
  % Packages and settings
  \usepackage{fontspec}
    \setmainfont{Charis SIL}
  \usepackage[margin=1.0in]{geometry}
  \usepackage[style=apa, backend=biber]{biblatex}
    \addbibresource{References.bib}

  % Document information
  \title{Summary of Sankoff \& Laberge (1978)}
  \author{Joshua McNeill}
  \date{\today}

  % New commands
  \newcommand{\orth}[1]{$\langle$#1$\rangle$}
  \newcommand{\lexi}[1]{\textit{#1}}

\begin{document}
  \maketitle
  \fullcitebib{sankoff_linguistic_1978}
  \section{The Linguistic Market}
    Objective: To develop a measure of social class based on a speaker's use of the ``legitimized language'', following Bourdieu, which is essential the standard language of a society.

    {\noindent
    Rationale: The way that speakers are placed into socioeconomic classes up to that time was not particularly motivated in Sankoff \& Laberge's view.
    }
    \begin{itemize}
      \item Often, speakers with similar wages were grouped together, though Sankoff \& Laberge argue that teachers and receptionists, for example, probably speak a more standard variety than others with whom they are grouped due to the environment in which they typically find themselves
    \end{itemize}

    In particular, Sankoff \& Laberge chose not to focus on occupations for social class.
    They provided several reasons for this.
    \begin{itemize}
      \item They were interested in individuals, many of whom did not even have occupations.
      \item Occupation classification is static, not easily showing behavior over time.
      \item Ranking the occupations would be arbitrary and subjective.
      \item Occupations do not allow for cross-cultural comparisons.
    \end{itemize}

  \section{The Judgements and Construction of the Index}
    Sankoff \& Laberge decided to use life histories as an indicator of social class instead of the typical socioeconomic factors.
    Speakers' life histories were written on pieces of paper that were then given to peers who were unacquainted with the speakers.
    These were peers in the sense that they were part of the same broader community.
    The task of these peers was to read the life histories and judge how important the legitimized language would be in the life of said person.
    The peers would then group the speakers, putting together those who seemed to have a similar legitimized language importance.
    They could come up with as many groups as they wanted as these groupings were later standardized on a scale from 0 to 1 where 0 represented those for whom the legitimized language was least important and 1 represented those for whom it was the most important.

    Sankoff \& Laberge then validated the resulting index using a dissimilarity matrix.
    Two different measures were used.
    \begin{itemize}
      \item The ratio of disagreements by pairs of judges of a speaker to total pairs
      \item The ratio of disagreements for a speaker to all disagreements
    \end{itemize}
    Both resulted in very low values, indicated the validity of their index.

  \section{The Index and Linguistic Behavior}
    The data for this study came from a corpus of Montreal French.
    Sankoff \& Laberge analyzed three lexical linguistic variables:
    \begin{itemize}
      \item \lexi{avoir} vs \lexi{être} in passé composé constructions
      \item \lexi{ce que} vs \lexi{qu'est-ce que} as relative pronouns without antecedents
      \item \lexi{on} vs \lexi{ils} for impersonal subject pronouns
    \end{itemize}

    Besides the linguistic market index, they looked at statistical associations with education, gender, and age, as well.
    These served as comparisons against the results of the index.
    For all three variables, the linguistic market index was easily the greatest predictor of variant use.
    Specifically, as a speaker's place in the index goes up, their usage on the standard variants goes up, those being \lexi{être}, \lexi{ce que}, and \lexi{on}, respectively.
    Sankoff \& Laberge took this to be very good evidence of the superiority of their linguistic market index to previously used measures of socioeconomic class.

  \section{Discussion}
    \begin{enumerate}
      \item Why are life histories so important?
      \item Would this index be usable in every society?
      \item To what extent are peers of a speaker still peers when they are unacquainted with a speaker?
    \end{enumerate}
\end{document}
