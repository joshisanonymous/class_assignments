\documentclass{article}
    % Packages and settings
    % \usepackage{acl2015}
    \usepackage{fontspec}
        \setmainfont{Charis SIL}
    \usepackage[style=apa, backend=biber]{biblatex}
        \addbibresource{References.bib}
    \usepackage{graphicx}
    \usepackage{listings}
        \lstset{basicstyle=\ttfamily,
                breaklines=true}
    \usepackage{hyperref}
        \hypersetup{colorlinks=true,
                    allcolors=blue}

    % Additional commands
    \newcommand{\orth}[1]{$\langle$#1$\rangle$}

    % Document information
    \title{Part-of-speech tagging for sociolinguistic interview data}
    \author{Joshua McNeill}
    \date{\today}

\begin{document}
  \maketitle
  \begin{abstract}
    Part of speech tagging systems have been a major area of natural language processing at least since the 1980s \parencite[e.g.,][]{garside_claws_1987}.
    However, much of the training and testing for part of speech tagging has been done using written language as opposed to spoken language.
    One major area in which spoken language differs from written language is in the presence of disfluences such as repetitions, hesitations, and repairs.
    As such, I propose tagging training data with disfluencies as part of the part of speech tags.
    For example, NNs that are within the reparandum would instead be labeled NN\_DIS.
    As a result, the tagger would potentially be able to identify disfluencies when used, and these disfluencies would not have to be normalized out of the data before tagging, which is often not what those working with natural speech want to do.

    This modification fo the training data is not expected to improve the accuracy of taggers for spoken language.
    Indeed, a few attempts have already been made to improve accuracy by incorporating prosodic features from the original audio \parencite[e.g.,][]{christodoulides_dismo:_2018}.
    However, the original audio is not always available, particularly when working with secondary data.
    Additionally, being able to tag disfluencies automatically without the need for audio would likely be far more efficient.

    This modification would potentially be a sign that disfluencies can be accurately identified in language even without prosodic information.
    To a degree, this is a linguistic hypothesis that can be tested here.
  \end{abstract}
  % \section{Introduction}
  %
  % \section{The Problem}
  %
  % \section{Methodology}
  %
  % \section{Conclusion}

  \printbibliography
\end{document}
