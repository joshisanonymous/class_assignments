\documentclass{article}
  % Packages and package settings
  \usepackage{fontspec}
    \setmainfont{Charis SIL}
  \usepackage{hyperref}
    \hypersetup{colorlinks=true,
                allcolors=blue}
  \usepackage{caption}
  \usepackage{graphicx}
    \graphicspath{{./figures/}}
  \usepackage{enumerate}

  % Title info
  \author{Joshua McNeill}
  \title{Homework 5}
  \date{\today}

  % Custom commands
  \newcommand{\lexi}[1]{\textit{#1}}
  \newcommand{\regfig}[2]{
    \begin{minipage}[t]{\linewidth}
      \captionof{figure}{#1}
      \label{fig:#2}
      \centering
      \includegraphics[scale=0.8]{#2.jpg}
    \end{minipage}
  }

\begin{document}
  \maketitle
  \begin{enumerate}[I]
    \item
    \begin{enumerate}
      \item[2] The content in the recordings \texttt{mollywillrun1} and \texttt{mollywillrun2} is identical: Both are `Molly will run in the rain.' However, the pitch patterns are not at all identical.
      \item[5]
      \begin{minipage}[t]{\linewidth}
        \captionof{table}{Average F0s of stressed words}
        \label{tab:avgf0}
        \centering
        \begin{tabular}{l | r r}
                & \multicolumn{2}{c}{F0 (Hz)} \\
          Word  & \texttt{mollywillrun1}  & \texttt{mollywillrun2} \\
          \hline
          Molly & 123                     & 124 \\
          run   & 123                     & 189 \\
          rain  & 108                     & 193
        \end{tabular}
      \end{minipage}
      \item[6]
      \regfig{Pitch contour for \texttt{mollywillrun1}}{mollywillrun1}
      \regfig{Pitch contour for \texttt{mollywillrun2}}{mollywillrun2}
      \item[7] Pitch helps signal the difference between statements and questions in English by increasing towards the end when the utterance is a question.
    \end{enumerate}
    \item
    \begin{enumerate}
      \item[11]
      \begin{minipage}[t]{\linewidth}
        \captionof{table}{Characteristics of stressed vs unstressed words}
        \label{tab:stress}
        \centering
        \begin{tabular}{l | r r r | r r r}
                & \multicolumn{3}{c}{\texttt{marywillwin1}} & \multicolumn{3}{c}{\texttt{marywillwin2}} \\
          Word  & dB  & F0 (Hz) & Dur. (ms)                 & dB  & F0 (Hz) & Dur. (ms) \\
          \hline
          Mary  & 70  & 123     & 433                       & 77  & 157     & 483 \\
          medal & 69  & 116     & 429                       & 66  & 107     & 404
        \end{tabular}
      \end{minipage}
      \item[12]
      \regfig{Intensity contour for \texttt{marywillwin1}}{marywillwin1}
      \regfig{Intensity contour for \texttt{marywillwin2}}{marywillwin2}
      \item[13] Amplitude, pitch and duration all increase to signal stress in English.
    \end{enumerate}
    \item
    \begin{enumerate}
      \item[15d]
      \begin{minipage}[t]{\linewidth}
        \captionof{table}{Lab vowel formants}
        \label{tab:vowels_lab}
        \centering
        \begin{tabular}{l l | r r}
          Word          & Phone & F1 (Hz) & F2 (Hz) \\
          \hline
          \lexi{bead}   & [i]   & 404     & 2490 \\
          \lexi{bid}    & [ɪ]   & 522     & 2131 \\
          \lexi{bade}   & [e]   & 437     & 2268 \\
          \lexi{bed}    & [ɛ]   & 592     & 1972 \\
          \lexi{bad}    & [æ]   & 653     & 1951 \\
          \lexi{booed}  & [u]   & 390     & 1694 \\
          \lexi{book}   & [ʊ]   & 588     & 1124 \\
          \lexi{boat}   & [o]   & 514     & 1515 \\
          \lexi{bought} & [ɔ]   & 613     & 1087 \\
          \lexi{body}   & [ɑ]   & 685     & 1282 \\
          \lexi{bud}    & [ʌ]   & 613     & 1185 \\
        \end{tabular}
      \end{minipage}
      \item[16]
      \regfig{Lab vowel plot}{vowels_lab}
      \item[18b]
      \begin{minipage}[t]{\linewidth}
        \captionof{table}{Josh vowel formants}
        \label{tab:vowels_josh}
        \centering
        \begin{tabular}{l l | r r}
          Word          & Phone & F1 (Hz) & F2 (Hz) \\
          \hline
          \lexi{bead}   & [i]   & 251     & 2296 \\
          \lexi{bid}    & [ɪ]   & 328     & 1918 \\
          \lexi{bade}   & [e]   & 369     & 2158 \\
          \lexi{bed}    & [ɛ]   & 503     & 1688 \\
          \lexi{bad}    & [æ]   & 606     & 1669 \\
          \lexi{booed}  & [u]   & 291     & 1495 \\
          \lexi{book}   & [ʊ]   & 517     & 1109 \\
          \lexi{boat}   & [o]   & 487     & 1184 \\
          \lexi{bought} & [ɔ]   & 620     & 1285 \\
          \lexi{body}   & [ɑ]   & 927     & 1224 \\
          \lexi{bud}    & [ʌ]   & 587     & 1146 \\
        \end{tabular}
      \end{minipage}
    \end{enumerate}
    \item[18d]
    \regfig{Josh vowel plot}{vowels_josh}
    \item[19] My vowel space is not unusual in any obvious way, except perhaps the presence of a very low [ɑ]. In fact, one of the most immediate differences between my own vowel space and that from the lab recording is that I have both higher high vowels and lower low vowels than the lab recording. Another notable difference is the position of the vowels relative to each other. For instance, my [o] is very close to my [ʌ], but the lab recording has a fronted [o]. Also, my [ɪ] is higher than my [e], but the lab recording has the opposite pattern.
  \end{enumerate}
\end{document}
