\documentclass{article}
  % Packages and settings
  \usepackage{fontspec}
    \setmainfont{Charis SIL}
  \usepackage[margin=1.0in]{geometry}
  \usepackage[style=apa, backend=biber]{biblatex}
    \addbibresource{References.bib}

  % Document information
  \title{Summary of Johnstone et al. (2006)}
  \author{Joshua McNeill}
  \date{15 October 2020}

  % New commands
  \newcommand{\orth}[1]{$\langle$#1$\rangle$}
  \newcommand{\lexi}[1]{\textit{#1}}
  \newcommand{\gloss}[1]{`#1'}

\begin{document}
  \maketitle
  \fullcitebib{johnstone_mobility_2006}
  \section{Introduction}
    The goal of \textcite{johnstone_mobility_2006} trace the history of how particular linguistic features in came to be noticed and a nameable language variety developed in the form of Pittsburghese.
    They do this using \citeauthor{silverstein_indexical_2003}'s (\citeyear{silverstein_indexical_2003}) indexicality framework wherein first order indexicals are variants associated with a % finish
    In particular, they argue that economic changes and globalization played an important role in this process.

    Their data comes from 1) newspaper articles about Pittsburgh speech published from 1910 to 2001, 2) sociolinguistic interviews, 3) experimental tasks, 4) historical research, and 5) participant observation.
    Specifically, though, they use discourse from five speakers who participated in a larger study of Pittsburgh.
    These five are from two white neighborhoods, one working class and the other a suburban area that ``boomed in the post-World War II years.''

    \textcite{johnstone_mobility_2006} describe languages and dialects as imagined ``cultural constructs'' with the result being that they are ``unstable'' in the face of contact with outsiders as those encounters make previously unnoticed local features visible via contrast.
    They argue that how this process played out for Pittsburghese involved the same mechanisms present in \citeauthor{agha_social_2003}'s (\citeyear{agha_social_2003}) study of RP.

  \section{The First Order}
    There are a number of phonological, lexical, and morphosyntactic features that are regional and found in Pittsburgh but that are not specific to Pittsburgh only, and these are the first order indexicals (e.g., the use of \lexi{yinz} \gloss{you (pl.)}).
    Before 1960 in particular, speakers were not aware that they used regionally distinctive features, as Dottie X. (b. 1930) explains for \lexi{yinz}.

  \section{Social Mobility and the Second Order}
    Whereas the two oldest participants were all but unaware of the distinctive features in their speech, the younger speakers interviewed with them were in fact aware due to growing up in an environment where there was greater social mobility.
    For example, Dr John K. does not even realize he uses monophthongal (aw) when describing this is incorrect speech, whereas Barb E. (b. 1957) is not only aware of these features but explicitly describes her own style shifting to accommodate her interlocutors.

  \section{Geographic Mobility and the Third Order}
    Much more geographic mobility was available to Pittsburghers after World War II.
    Some could afford to vacation at east coast beaches, and some traveled in the military.
    By the 1960s, Pittsburghers started identifying with their locale, whereas their grandparents might have identified with their home countries or religions.
    This process was pushed even further in the 1980s when economic pressures led to Pittsburghers migrating out of the city where they were told the spoke funny.

    In public discourse, this process began with describing Pittsburghese disparagingly.
    However, the dialect started to be legitimized in the 1960s when dialectologists Parslow started to be cited.

    In this context, use of local features becomes performative.
    For example, Jessica H. (b. 1979) is a local who is aware of Pittsburghese enough to use its features but does not have these features.

  \section{Discussion}
    \begin{enumerate}
      \item Johnstone et al. describe speakers in the early 20th century as simply not having access to non-local variants, but they also describe (aw) monophthongization as being more prevalent in working class and male speakers. This implies that some other variant must have existed, too, in which case, why might this have not been considered accessible?
      \item Why is the topic of stigma not brought up again in the third order section of the article?
      \item How does the case of Pittsburghese compare with southern speech?
    \end{enumerate}
\end{document}
